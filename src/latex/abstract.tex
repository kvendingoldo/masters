% автореферат
\documentclass[master, och, autoref, times]{sty/SCWorks}

\usepackage[T2A]{fontenc}
\usepackage[utf8]{inputenc}
\usepackage{graphicx}
\usepackage{float}

\usepackage[sort,compress]{cite}
\usepackage{amsmath}
\usepackage{amssymb}
\usepackage{amsthm}
\usepackage{longtable}
\usepackage{array}
\usepackage[english,russian]{babel}
\usepackage{amsfonts}
\usepackage{commath}
\usepackage{amsthm}

\usepackage[colorlinks=true]{hyperref}

\usepackage{listings}

\usepackage[ruled,vlined,linesnumbered,resetcount,algosection]{algorithm2e}

\usepackage{algpseudocode}

\usepackage{longtable,rotating}
\usepackage{threeparttable}

% для кода
\usepackage{fancyvrb}
\DefineShortVerb{\|}

% для картинок
\usepackage{chngcntr}
\counterwithin{figure}{section}

\theoremstyle{plain}
\newtheorem{thethm}{Теорема}[section]
\newtheorem{lemma}{Лемма}[section]
\newtheorem{note}{Замечание}[section]
\newtheorem{proposition}{Предложение}[section]
\newtheorem{exmp}{Пример}[section]
\newtheorem{problem}{Проблема}[section]

\theoremstyle{definition}
\newtheorem{defn}{Определение}[section]

\numberwithin{equation}{section}

\SetKwInput{KwData}{Исходные параметры}
\SetKwInput{KwResult}{Результат}
\SetKwInput{KwIn}{Входные данные}
\SetKwInput{KwOut}{Выходные данные}
\SetKwIF{If}{ElseIf}{Else}{если}{тогда}{иначе если}{иначе}{конец условия}
\SetKwFor{While}{до тех пор, пока}{выполнять}{конец цикла}
\SetKw{KwTo}{от}
\SetKw{KwRet}{возвратить}
\SetKw{Return}{возвратить}
\SetKwBlock{Begin}{начало блока}{конец блока}
\SetKwSwitch{Switch}{Case}{Other}{Проверить значение}{и выполнить}{вариант}{в противном случае}{конец варианта}{конец проверки значений}
\SetKwFor{For}{цикл}{выполнять}{конец цикла}
\SetKwFor{ForEach}{для каждого}{выполнять}{конец цикла}
\SetKwRepeat{Repeat}{повторять}{до тех пор, пока}
\SetAlgorithmName{Алгоритм}{алгоритм}{Список алгоритмов}

\algrenewcommand\algorithmicfunction{\textbf{Функция}}
\algrenewcommand\algorithmicprocedure{\textbf{Процедура}}
\algrenewtext{EndFunction}{\textbf{Конец функции}}
\algrenewtext{EndProcedure}{\textbf{Конец процедуры}}

% новая секция
\newcommand\summary[1][]{\starsection{КРАТКОЕ СОДЕРЖАНИЕ РАБОТЫ}}




\begin{document}

% Кафедра (в родительном падеже)
\chair{дискретной математики}
% Тема работы
\title{Приложение $p$-адической арифметики к задачам компьютерной алгебры}
% Курс
\course{2}
% Группа
\group{271}
% Специальность/направление код - наименование
\napravlenie{09.04.01 "--- Информатика и вычислительная техника}

% Фамилия, имя, отчество в родительном падеже
\author{Шарова Александра Вадимовича}
% Заведующий кафедрой
\chtitle{к.\,ф.-м.\,н., доцент} % степень, звание
\chname{Л.\,Б.\,Тяпаев}
%Научный руководитель (для реферата преподаватель проверяющий работу)
\satitle{к.\,ф.-м.\,н., доцент} %должность, степень, звание
\saname{Л.\,Б.\,Тяпаев}
% Руководитель практики от организации (только для практики,
% для остальных типов работ не используется)
\patitle{к.\,ф.-м.\,н., доцент}
\paname{Д.\,Ю.\,Петров}

% Год выполнения отчета
\date{2020}

\maketitle

\intro
Компьютерная алгебра -- раздел математики, образовавшийся на стыке алгебры, вычислительных методов и символьных вычислений. Для данного раздела, как и для любого другого, сочетающего в себе различные области науки, трудно определить конкретные границы.

Термин "компьютерная алгебра"\ возник как синоним для терминов "символьные вычисления"\ , "аналитические вычисления"\ и "аналитические преобразования". На французском языке данный термин и в настоящее время означает "формальные вычисления".

Когда говорят о вычислительных методах, считается, что производимые вычисления выполняются в поле вещественных или, в ряде задач, комплексных чисел. В действительности же всякая программа для ЭВМ имеет дело только с конечным набором целых чисел, поскольку только такие числа могут быть представлены в компьютере. Для хранения целого числа отводится обычно 16 или 32 двоичных символа (бита), для вещественного - 32 или 64 бита. Это множество не замкнуто относительно арифметических операций, что может выражаться в различных переполнениях, например, при умножении достаточно больших чисел (изучаемых гугологией) или при делении на маленькое ($10^{-10}$ и меньше) число. Еще более  существенной особенностью вычислительной математики является то, что арифметические операции над этими числами, производимые компьютером, отличаются от арифметических операций в поле рациональных чисел. Более того, для компьютерных операций не выполняются основные аксиомы поля (ассоциативности и дистрибутивности). В основном в компьютерной алгебре вычисления производятся точно, без округления. Тем не менее в данном разделе математики рассматриваются и задачи, требующие приближенного решения, например, задачи из таких областей, как механика и термодинамика.

На самом деле при работе с рациональными числами вычисления происходят с приближенными значениями, которые представляют собой десятичные (или двоичные, при использовании компьютера) дроби с фиксированным числом значащих цифр.  Как правило, данные вычисления являются недостаточно точными или затратными с точки зрения вычислительных ресурсов и времени.
Из курса математического анализа известно, что поле вещественных чисел $\mathbb{R}$ можно определить как пополнение поля рациональных чисел $\mathbb{Q}$ по архимедовой метрике, когда расстояние между двумя рациональными числами определяется как модуль их разности. В математике, в частности в теории чисел, рассматриваются также другие метрики поля рациональных чисел, так называемые $p$-адические. При пополнении поля $\mathbb{Q}$ по $p$-адической метрике получается поле $p$-адических чисел, которые играют значительную роль в теории чисел.


Целью данной магистерской выпускной квалификационной работы является создание программного комплекса на языке программирования Python для работы с наиболее распространенными объектами компьютерной алгебры, которые будут использовать $p$-адическую арифметику над $\mathbb{Q}_p$, а также произведение сравнения времени вычислений между полученным программным комплексом для работы с $p$-адическими числами и $p$-адической арифметикой и стандартными математическими пакетами, которые наиболее часто используются для решения прикладных задач в ЯП Python. Сравнение будет производится с помощью ряда методов, каждый из которых будет подробно описан в случае конкретной задачи.

Актуальность данной работы обусловлена исследованием как подходов использования $p$-адической арифметики в рамках вычислительных методов и компьютерной алгебры, так и некоторых теоретических особенностей $p$-адической арифметики, которые применяются к вычислениям, производимым на компьютере. Данные подходы и особенности позволяют ускорить вычисления за счет выбора правильного основания $p$-адического числа или же  позволяют получить более точный результат за счет того, что $p$-адическая арифметика не накапливает арифметическую ошибку в отличие от классической арифметики (что является очень важным для многих областей прикладной математики и физики, где важны точность и время производимых вычислений).

В данной работе будут представлены основные определения и понятия $p$-адической арифметики и $p$-адического анализа, а также все необходимые математические определения и теоремы, используемые для рассматриваемых прикладных задач. Будет дано описание представления $p$-адических чисел в виде кода Гензеля, а затем представлен и описан программный комплекс для работы с $p$-адическими числами, разработанный с использованием  ЯП Python. Также будет рассмотрен важный вопрос о параллелизации представленных алгоритмов, которые используют $p$-адические вычисления. Кроме того, будет представлена программная реализация, которая позволяет эффективно использовать ресурсы процессора при работе с представленным программным комплексом. В качестве прикладных примеров для тестирования программного комплекса будут использоваться классические задачи компьютерной алгебры и численных методов, такие как нахождение решения СЛАУ, ОДУ, вычисление собственных чисел и собственных значений векторов матрицы, вычисление матричной экспоненты. Для наглядности все тесты производительности будут выполнены как для однопоточной, так и для многопоточной реализации программного комплекса, результат будет представлен в виде графиков и диаграмм с сопутствующими описаниями полученных результатов для всех проведенных тестов.



\summary 
фыв




\conclusion
В данной магистерской выпускной квалификационной работе представлен программный комплекс, разработанный с использованием ЯП Python и предназначенный для осуществления операций с наиболее распространенными объектами компьютерной алгебры при использовании $p$-адической арифметики над полем $\mathbb{Q}_p$. Представлены и описаны типы данных и алгоритмы работы с $p$-адическими числами как для однопоточного, так и для многопоточного случая.

Разработанный программный комплекс для работы с $p$-адическими числами не имеет аналогов для ЯП Python в открытом доступе, что делает его актуальным в случае необходимости применения $p$-адической арифметики при решении прикладных задачах с использованием данного ЯП.

Для тестирования программного комплекса было проведено решение таких популярных прикладных задач, как нахождение решения обыкновенного дифференциального уравнения, нахождение решения СЛАУ, а также вычисление собственных значений и векторов матрицы и вычисление матричной экспоненты. При проведении тестов  основным рассматриваемым параметром было время вычисления для каждой из рассматриваемых задач, а на его основании производилось сравнение со временем, за которое стандартные библиотеки находили решение.

Для всех тестов были произведены замеры производительности работы как для однопоточного, так и для многопоточного варианта вычислений, а также приведены графики и диаграммы, отражающие полученные результаты. Тесты производительности многопоточной библиотеки показали, что что параллельная $p$-адическая арифметика сравнима с классическими методами, но, несмотря на это, является лучшей с той точки зрения, что во время вычислений не накапливает арифметическую ошибку. Очевидно, что классические методы тоже могут дать сколь угодно точный результат, но при этом время вычислений будет расти в разы, когда как в случае с $p$-адической арифметикой время будет константным.

Результат работы позволяет сделать вывод, что $p$-адическую арифметику выгодно использовать для ряда прикладных задач, которые решаются с помощью ЯП Python на распределенных кластерах или в облачной среде, так как время вычисления не уступает классическим методам, а в случае использования чисел из $\mathbb{Q}_2$ и $\mathbb{Q}_3$ даже превосходит их.

// TODO
Отдельные части магистерской работы были опубликованы на конференции:
TODO

Основные источники информации:
TODO





\end{document}