% автореферат
\documentclass[master, och, autoref, times]{sty/SCWorks}

\usepackage[T2A]{fontenc}
\usepackage[utf8]{inputenc}
\usepackage{graphicx}
\usepackage{float}

\usepackage[sort,compress]{cite}
\usepackage{amsmath}
\usepackage{amssymb}
\usepackage{amsthm}
\usepackage{longtable}
\usepackage{array}
\usepackage[english,russian]{babel}
\usepackage{amsfonts}
\usepackage{commath}
\usepackage{amsthm}

\usepackage[colorlinks=true]{hyperref}

\usepackage{listings}

\usepackage[ruled,vlined,linesnumbered,resetcount,algosection]{algorithm2e}

\usepackage{algpseudocode}

\usepackage{longtable,rotating}
\usepackage{threeparttable}

% для картинок
\usepackage{chngcntr}
\counterwithin{figure}{section}

\theoremstyle{plain}
\newtheorem{thethm}{Теорема}[section]
\newtheorem{lemma}{Лемма}[section]
\newtheorem{note}{Замечание}[section]
\newtheorem{proposition}{Предложение}[section]
\newtheorem{exmp}{Пример}[section]
\newtheorem{problem}{Проблема}[section]

\theoremstyle{definition}
\newtheorem{defn}{Определение}[section]

\numberwithin{equation}{section}


% для кода
\usepackage{fancyvrb}
\DefineShortVerb{\|}

% новая секция
\newcommand\summary[1][]{\starsection{КРАТКОЕ СОДЕРЖАНИЕ РАБОТЫ}}
% загрузка модифицированного файла для именения шапки библиографии
\usepackage{sty/misc}
% загрузка нормального шрифта, позволяющего делать textbf
\renewcommand{\rmdefault}{cmr}


\begin{document}

% Кафедра (в родительном падеже)
\chair{дискретной математики}
% Тема работы
\title{Приложение $p$-адической арифметики к задачам компьютерной алгебры}
% Курс
\course{2}
% Группа
\group{271}
% Специальность/направление код - наименование
\napravlenie{09.04.01 "--- Информатика и вычислительная техника}

% Фамилия, имя, отчество в родительном падеже
\author{Шарова Александра Вадимовича}
% Заведующий кафедрой
\chtitle{к.\,ф.-м.\,н., доцент} % степень, звание
\chname{Л.\,Б.\,Тяпаев}
%Научный руководитель (для реферата преподаватель проверяющий работу)
\satitle{к.\,ф.-м.\,н., доцент} %должность, степень, звание
\saname{Л.\,Б.\,Тяпаев}
% Руководитель практики от организации (только для практики,
% для остальных типов работ не используется)
\patitle{к.\,ф.-м.\,н., доцент}
\paname{Д.\,Ю.\,Петров}

% Год выполнения отчета
\date{2020}

\maketitle

\intro

% -- актульность -- %
\textbf{Актуальность темы}.

Компьютерная алгебра -- раздел математики, образовавшийся на стыке алгебры, вычислительных методов и символьных вычислений. Для данного раздела, как и для любого другого, сочетающего в себе различные области науки, трудно определить конкретные границы.

Термин "компьютерная алгебра"\ возник как синоним для терминов "символьные вычисления"\ , "аналитические вычисления"\ и "аналитические преобразования". На французском языке данный термин и в настоящее время означает "формальные вычисления".

Когда говорят о вычислительных методах, считается, что производимые вычисления выполняются в поле вещественных или, в ряде задач, комплексных чисел. В действительности же всякая программа для ЭВМ имеет дело только с конечным набором целых чисел, поскольку только такие числа могут быть представлены в компьютере. Для хранения целого числа отводится обычно 16 или 32 двоичных символа (бита), для вещественного - 32 или 64 бита. Это множество не замкнуто относительно арифметических операций, что может выражаться в различных переполнениях, например, при умножении достаточно больших чисел (изучаемых гугологией) или при делении на маленькое ($10^{-10}$ и меньше) число. Еще более  существенной особенностью вычислительной математики является то, что арифметические операции над этими числами, производимые компьютером, отличаются от арифметических операций в поле рациональных чисел. Более того, для компьютерных операций не выполняются основные аксиомы поля (ассоциативности и дистрибутивности). В основном в компьютерной алгебре вычисления производятся точно, без округления. Тем не менее в данном разделе математики рассматриваются и задачи, требующие приближенного решения, например, задачи из таких областей, как механика и термодинамика.

На самом деле при работе с рациональными числами вычисления происходят с приближенными значениями, которые представляют собой десятичные (или двоичные, при использовании компьютера) дроби с фиксированным числом значащих цифр.  Как правило, данные вычисления являются недостаточно точными или затратными с точки зрения вычислительных ресурсов и времени.
Из курса математического анализа известно, что поле вещественных чисел $\mathbb{R}$ можно определить как пополнение поля рациональных чисел $\mathbb{Q}$ по архимедовой метрике, когда расстояние между двумя рациональными числами определяется как модуль их разности. В математике, в частности в теории чисел, рассматриваются также другие метрики поля рациональных чисел, так называемые $p$-адические. При пополнении поля $\mathbb{Q}$ по $p$-адической метрике получается поле $p$-адических чисел, которые играют значительную роль в теории чисел.

Актуальность данной работы обусловлена исследованием как подходов использования $p$-адической арифметики в рамках вычислительных методов и компьютерной алгебры, так и некоторых теоретических особенностей $p$-адической арифметики, которые применяются к вычислениям, производимым на компьютере. Данные подходы и особенности позволяют ускорить вычисления за счет выбора правильного основания $p$-адического числа или же  позволяют получить более точный результат за счет того, что $p$-адическая арифметика не накапливает арифметическую ошибку в отличие от классической арифметики (что является очень важным для многих областей прикладной математики и физики, где важны точность и время производимых вычислений).


% -- цель -- %
\textbf{Цель магистерской работы}.

\begin{enumerate}
	\item Дать алгоритмическое описание $p$-адической арифметики и реализовать ее на языке Python;
	\item Составить библиотеку, которая будет базироваться на уже представленной реализации $p$-адической арифметики и будет содержать следующий функционал:  
	\begin{itemize}
    \item базовые математические операции над матрицами и векторами;
    \item вычисление определителя матрицы;
    \item алгоритм для нахождения решения СЛАУ;
    \end{itemize}
    \item Привести примеры применения реализованной однопоточной библиотеки. Составить следующие программы:
	\begin{itemize}
    \item для вычисления матричной экспоненты $e^{Ax}$;
    \item для нахождения решения СЛАУ методом Крамера и Гаусса;
    \item для вычисления собственных значений и собственных чисел \mbox{матрицы};
    \item для нахождения решения ОДУ;
    \end{itemize}
    \item Реализовать многопоточный вариант p-адической арифметики и \mbox{провести} сравнение с однопоточным \mbox{вариантом} на примерах решения прикладных задач;
    \item Провести сравнение многопоточной библиотеки p-адической \mbox{арифметики} с классическими методами на следующих задачах:
    \begin{itemize}
    \item нахождение решения СЛАУ;
    \item нахождение решения ОДУ;    
 	\item вычисление матричной экспоненты;
 	\item вычисление собственных значений и собственных векторов симметричной матрицы.
    \end{itemize}
\end{enumerate}


% -- методологические основы -- %
\textbf{Методологические основы} $p$-адической арифметики представлены в работах В.С. Анашина, А. Ю. Хренникова, В.С. Владимирова, И.В. Воловича, Е.И. Зеленова и С. В. Козырева. Среди зарубежных учёных можно выделить таких ученых как X.Li, C. Lu, A. Sjogren, A. Miola, R. Gregory а также I.Ojalvo и M. Newman.


% -- значимость -- %
\textbf{Теоретическая и практическая значимость магистерской работы}. Проблема увеличения точности и уменьшения времени вычислений на ЭВМ стояла всегда и ученые до сих пор продолжают изобретать все более совершенные алгоритмы для решения популярных прикладных задач физики и математики. Использование $p$-адической арифметики является новых подходом для вычислений производимых на компьютере и позволяет достичь лучших результатов в ряде прикладных задач как по времени выполнения алгоритмов, так и в точности получаемых вычислений. Нельзя переоценить важность этих результатов особенно в прикладных областях науки которые оперируют огромными массивами входящих данных и требуют сложных вычислений производимых с высокой точностью.


% -- структура -- %
\textbf{Структура и объем работы}.
Магистерская работа состоит из введения, 5 разделов, заключения, списка использованных источников и 2 приложений. Общий объем работы -- 99 страниц, из них 63 страницы -- основное содержание, включая 13 рисунков и список использованных источников \mbox{информации -- 40} наименований.


\summary 
\textbf{Первый раздел <<$p$-адические числа>>} посвящен теоретическим основам $p$-адической арифметики и $p$-адического анализа. В данном разделе приведены основные теоретические сведения такие как определение $p$-адической нормы, пространства $p$-адических чисел $\mathbb{Q}_p$ и его свойств, а также ряд определений и лемм имеющих отношение к $p$-адическая дифференцируемости.


\textbf{Второй раздел <<Представления $p$-адических чисел и арифметические операции>>} посвящен описанию представлению рациональных чисел в $p$-адической форме. В данном разделе введены такие теоретические сведения как определение канонической формы $p$-адического числа и способ вычисления $p$-адического представления для некоторого рационального числа $\alpha$. Кроме того, рассмотрены основные арифметические операции для $p$-адических чисел и приведены примеры. Помимо этого, введены понятия кода и псевдокода Гензеля которые позволяют представлять $p$-адические числа в компьютере. Для кода Гензеля также рассмотрены такие основные арифметические операции как сложение, вычитание, умножение и деление.

\textbf{Третий раздел <<Разработка однопоточной библиотеки для работы с $p$-адической арифметикой>>} посвящен описанию разработки однопоточной библиотеки на ЯП Python позволяющей представлять $p$-адические числа в компьютере, а также производить базовые арифметические операции над ними. В данном разделе дано описание всех методов для разработанных классов |Padic| и |Matrix|, описаны основные особенности разработанных типов данных. Для наглядности приведены базовые примеры использования библиотеки. Для тестирования полученной библиотеки были решены задачи нахождения решения СЛАУ и вычисления определителя матрицы. Для данных прикладных задач были произведены замеры производительности на основании которых было произведено сравнение с аналогичными методами из популярных математических библиотек для ЯП Python. 


\textbf{Четвертый раздел <<Разработка многопоточной библиотеки для работы с $p$-адической арифметикой>>} посвящен описанию разработки многопоточной библиотеки. В данном разделе приведено описание многопоточной $p$-адической арифметики и приведен алгоритм для кодирования и декодирования $p$-адических чисел. Кроме этого приведены многопоточные алгоритмы для нахождения решения СЛАУ и вычисления собственных чисел и собственных векторов матрицы.

\textbf{Пятый раздел <<Сравнение производительности классических и $p$-адических методов на примере прикладных задач>>} посвящен сравнению производительности $p$-адических и классических методов для таких задач, как: нахождение решения СЛАУ, вычисление собственных чисел и собственных значений матрицы, нахождение решения ОДУ, вычисление матричной экспоненты $e^{At}$. Для всех задач приведены описания экспериментов, сделаны необходимые замеры времени вычисления, а также приведены диаграммы на основании которых сделаны выводы о полученных результатах.


\conclusion

В данной магистерской выпускной квалификационной работе представлен программный комплекс, разработанный с использованием ЯП Python и предназначенный для осуществления операций с наиболее распространенными объектами компьютерной алгебры при использовании $p$-адической арифметики над полем $\mathbb{Q}_p$. Представлены и описаны типы данных и алгоритмы работы с $p$-адическими числами как для однопоточного, так и для многопоточного случая.

Разработанный программный комплекс для работы с $p$-адическими числами не имеет аналогов для ЯП Python в открытом доступе, что делает его актуальным в случае необходимости применения $p$-адической арифметики при решении прикладных задачах с использованием данного ЯП.

Для тестирования программного комплекса было проведено решение таких популярных прикладных задач, как нахождение решения обыкновенного дифференциального уравнения, нахождение решения СЛАУ, а также вычисление собственных значений и векторов матрицы и вычисление матричной экспоненты. При проведении тестов  основным рассматриваемым параметром было время вычисления для каждой из рассматриваемых задач, а на его основании производилось сравнение со временем, за которое стандартные библиотеки находили решение.

Для всех тестов были произведены замеры производительности работы как для однопоточного, так и для многопоточного варианта вычислений, а также приведены графики и диаграммы, отражающие полученные результаты. Тесты производительности многопоточной библиотеки показали, что что параллельная $p$-адическая арифметика сравнима с классическими методами, но, несмотря на это, является лучшей с той точки зрения, что во время вычислений не накапливает арифметическую ошибку. Очевидно, что классические методы тоже могут дать сколь угодно точный результат, но при этом время вычислений будет расти в разы, когда как в случае с $p$-адической арифметикой время будет константным.

Результат работы позволяет сделать вывод, что $p$-адическую арифметику выгодно использовать для ряда прикладных задач, которые решаются с помощью ЯП Python на распределенных кластерах или в облачной среде, так как время вычисления не уступает классическим методам, а в случае использования чисел из $\mathbb{Q}_2$ и $\mathbb{Q}_3$ даже превосходит их.


\textbf{Отдельные части магистерской работы были опубликованы на студенческой научной конференции 24 апреля 2020 года} где был представлен доклад о возможностях использования $p$-адической арифметики в ЭВМ, где были рассмотрены в том числе вопросы производительности на примере задачи вычисления матричной экспоненты.

\nocite{*}
\bibliographystyle{biblio/ugost}
\bibliography{biblio/basic}





\end{document}