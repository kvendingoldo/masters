% задание на магистерскую работу
\documentclass[master, och, times, assignment]{sty/SCWorks}

\usepackage[T2A]{fontenc}
\usepackage[utf8]{inputenc}
\usepackage{graphicx}
\usepackage{float}

\usepackage[sort,compress]{cite}
\usepackage{amsmath}
\usepackage{amssymb}
\usepackage{amsthm}
\usepackage{longtable}
\usepackage{array}
\usepackage[english,russian]{babel}
\usepackage{amsfonts}
\usepackage{commath}
\usepackage{amsthm}

\usepackage[colorlinks=true]{hyperref}

\usepackage{listings}

\usepackage[ruled,vlined,linesnumbered,resetcount,algosection]{algorithm2e}

\usepackage{algpseudocode}

\usepackage{longtable,rotating}
\usepackage{threeparttable}

% загрузка нормального шрифта, позволяющего делать textbf
\renewcommand{\rmdefault}{cmr}

\begin{document}

% Кафедра (в родительном падеже)
\chair{дискретной математики и информационных технологий}
% Тема работы
\title{asd}
% Курс
\course{2}
% Группа
\group{271}
% Специальность/направление код - наименование
\napravlenie{09.04.01 "--- Информатика и вычислительная техника}
% Тема работы
\title{Приложение $p$-адической арифметики к задачам компьютерной алгебры}
% Фамилия, имя, отчество в родительном падеже
\author{Шарова Александра Вадимовича}
\myname{Шаров А.В.}

% Заведующий кафедрой
\chtitle{к.\,ф.-м.\,н., доцент} % степень, звание
\chname{Л.\,Б.\,Тяпаев}
%Научный руководитель (для реферата преподаватель проверяющий работу)
\satitle{к.\,ф.-м.\,н., доцент} %должность, степень, звание
\saname{Л.\,Б.\,Тяпаев}
% Руководитель практики от организации (только для практики,
% для остальных типов работ не используется)
%\patitle{к.\,ф.-м.\,н., доцент}
%\paname{Л.\,Б.\,Тяпаев}
\spectyperod{10}

\satitle{к.\,ф.-м.\,н., доцент}
\saname{Л.\,Б.\,Тяпаев}

% протокол
\protnum{3}
\protdate{10.06.2019}

% секретарь
\secrname{Ульянова А.А.}

\practStart{09.10.2018}
\practFinish{10.06.2019}


% Год выполнения отчета
\date{2020}

\maketitle

Общая постановка задачи: создание программного комплекса на языке программирования Python для работы с наиболее распространенными объектами компьютерной алгебры которые будут использовать $p$-адическую арифметику над полем $\mathbb{Q}_p$, а также произведение сравнения времени вычислений между полученным программным комплексом для работы с $p$-адическими числами и $p$-адической арифметикой и стандартными математическими пакетами, которые наиболее часто используются для решения прикладных задач в ЯП Python. Для решения этой задачи необходимо:

\begin{enumerate}
	\item дать алгоритмическое описание $p$-адической арифметики и реализовать ее на языке Python;
	\item составить библиотеку, которая будет базироваться на уже представленной реализации $p$-адической арифметики и будет содержать следующий функционал:
	\begin{itemize}
    \item базовые математические операции над матрицами и векторами;
    \item вычисление определителя матрицы;
    \item алгоритм для нахождения решения СЛАУ;
    \end{itemize}
    \item привести примеры применения реализованной однопоточной библиотеки. Составить следующие программы:
	\begin{itemize}
    \item для вычисления матричной экспоненты $e^{Ax}$;
    \item для нахождения решения СЛАУ методом Крамера и Гаусса;
    \item для вычисления собственных значений и собственных чисел \mbox{матрицы};
    \item для нахождения решения ОДУ;
    \end{itemize}
    \item Реализовать многопоточный вариант p-адической арифметики и \mbox{провести} сравнение с однопоточным \mbox{вариантом} на примерах решения прикладных задач;
    \item провести сравнение многопоточной библиотеки p-адической \mbox{арифметики} с классическими методами на следующих задачах:
    \begin{itemize}
    \item нахождение решения СЛАУ;
    \item нахождение решения ОДУ;
 	\item вычисление матричной экспоненты;
 	\item вычисление собственных значений и собственных векторов симметричной матрицы.
    \end{itemize}
\end{enumerate}


В теоретической части работы необходимо описать используемые технологии и привести все необходимые предварительные сведения для \mbox{практической} реализации, а кроме того привести все необходимые \mbox{выкладки} для рассматриваемых формул, алгоритмов и теорем. В экспериментальной части работы необходимо описать разработанный программный продукт, продемонстрировать примеры его тестовых запусков и сделать соответствующие выводы на основании полученных результатов.

\signatureline

\end{document}
