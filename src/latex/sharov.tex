\documentclass[master, och, diploma, times]{sty/SCWorks}

\usepackage[T2A]{fontenc}
\usepackage[utf8]{inputenc}
\usepackage{graphicx}

\usepackage[sort,compress]{cite}
\usepackage{amsmath}
\usepackage{amssymb}
\usepackage{amsthm}
\usepackage{fancyvrb}
\usepackage{longtable}
\usepackage{array}
\usepackage[english,russian]{babel}
\usepackage{amsfonts}
\usepackage{commath}
\usepackage{amsthm}

\usepackage[colorlinks=true]{hyperref}

% для кода
\usepackage{fancyvrb}
\DefineShortVerb{\|}

\theoremstyle{plain}
\newtheorem{thethm}{Теорема}
\newtheorem{lemma}{Лемма}
\newtheorem{note}{Замечание}
\newtheorem{proposition}{Предложение}[section]
\newtheorem{exmp}{Пример}[section]

\theoremstyle{definition}
\newtheorem{defn}{Определение}[section]


\begin{document}

% Кафедра (в родительном падеже)
\chair{дискретной математики}
% Тема работы
\title{Приложение p-адической арифметики к задачам компьютерной алгебры}
% Курс
\course{2}
% Группа
\group{271}
% Специальность/направление код - наименование
\napravlenie{09.03.01 "--- Информатика и вычислительная техника}

% Фамилия, имя, отчество в родительном падеже
\author{Шарова Александра Вадимовича}
% Заведующий кафедрой
\chtitle{к.\,ф.-м.\,н.} % степень, звание
\chname{Л.\,Б.\,Тяпаев}
%Научный руководитель (для реферата преподаватель проверяющий работу)
\satitle{к.\,ф.-м.\,н.} %должность, степень, звание
\saname{Л.\,Б.\,Тяпаев}
% Руководитель практики от организации (только для практики,
% для остальных типов работ не используется)
\patitle{к.\,ф.-м.\,н., доцент}
\paname{Д.\,Ю.\,Петров}

% Год выполнения отчета
\date{2020}

\maketitle
\tableofcontents

\abbreviations
\begin{enumerate}
	\item $\mathbb {Z}$ -- кольцо целых рациональных чисел
	\item $\mathbb {Z}_{+}$ -- множество натуральных чисел $\mathbb {N}$
	\item ${N}_0=\{0,1,\dots\}$
	\item $\mathbb {P}$ -- множество простых чисел
	\item Сепарабельное пространство -- топологическое пространство, в котором можно выделить счётное всюду плотное подмножество
	\item Плотное множество -- подмножество пространства, точками которого можно сколь угодно хорошо приблизить любую точку объемлющего пространства.
	\item Счётное множество -- бесконечное множество, элементы которого возможно пронумеровать натуральными числами.
	\item Хаусдорфово пространство — топологическое пространство, удовлетворяющее сильной аксиоме отделимости $T_2$.
	\item Множество из $\mathbb {R}^n$ называется компактом, если из любой последовательности его точек можно выделить сходящуюся подпоследовательность, предел которой принадлежит этому множеству.
	\item Локально компактное пространство — топологическое пространство, у каждой точки которого существует открытая окрестность, замыкание которой компактно.
	\item Размерностью полного метрического пространства $X$ называется наименьшее целое число $n$ такое, что для любого покрытия пространства $X$ существуют вписанное в него подпокрытие кратности $n+1$.
\end{enumerate}

\intro
тут будет введение

\section{p-адические числа}

\subsection{p-адическая норма}

\begin{defn}
Пусть $M$ - некоторое непустое множество, и пусть \linebreak ${d: M \times M \rightarrow \mathbb {R}_{\ge0}}$ -- функция двух переменных, определенная на этом множестве и принимающая значения во множестве действительных неотрицательных чисел. Функция $d$ называется метрикой (а множество $M$ -- метрическим пространством), если $d$ удоволетворяет трем условиям:

\begin{enumerate} 
	\item Для каждой пары $a, b \in M$ справедливо: $d(a, b)=0$ тогда и только тогда, когда $a=b$.
	\item Для каждой пары $a, b \in M$ справедливо равенство $d(a, b) = d(b, a)$.
	\item Для каждой тройки $a, b, c \in M$ справедливо неравенство $d(a, b) \le d(a, c) + d(c, b)$.
\end{enumerate}
\end{defn}

\begin{exmp}
Множество $\mathbb {R}$ всех действительных чисел есть метрическое пространство с метрикой $d(a, b)= \abs{a-b}$, где $\abs{.}$ есть абсолютная величина.
\end{exmp}


\begin{defn}
Функция $\norm{.}$, определенная на произвольном коммутативном кольце R и принимающая значения в $\mathbb {R}_{\ge 0}$ называется нормой (также, абсолютной величиной), если она удовлетворяет следующим условиям:

\begin{enumerate} 
	\item Для любого $a \in R$ справедливо, что $\norm{a}=0$ тогда и только тогда, когда $a=0$.
	\item Для каждой пары $a, b \in R$ справедливо равенство $\norm{a \cdot b} = \norm{a} \cdot \norm{b}$.
	\item Для каждой пары $a, b \in R$ справедливо неравенство треугольника: $\norm{a + b} \le \norm{a} + \norm{b}$
\end{enumerate}
\end{defn}

Из определения следует, что если положить $d(a, b)= \norm{a - b}$, то фактически будет задана метрика $d$ на кольце $R$. Данная метрика называется метрикой, индуцированной нормой $\norm{.}$.

\begin{defn}
Пусть $p \in \mathbb {P}$ -- некоторое простое число. В поле $\mathbb {Q}$ введем другую норму $\norm{.}_p$ по правилу:

\begin{enumerate} 
	\item $\norm{0}_p = 0$,
	\item $\norm{n}_p = p ^ {-ord_pn}$,
\end{enumerate}

\noindent где $n > 0$ некоторое натуральное число, а $ord_pn$ показатель степени, в которой число $p$ входит в это произведение. В этом случае норма $\norm{.}_p$ называется $p$-адической нормой.
\end{defn}

Норма $\norm{.}_p$  удовлетворяет всеми характерными свойствами нормы даже в более сильной форме, а именно:

\begin{enumerate} 
	\item $\norm{x}_p \ge 0$, причем $\norm{x}_p = 0$ если $x = 0$.
	\item $\norm{xy}_p = \norm{x}_p \cdot \norm{y}_p$.
	\item $\norm{x + y}_p \le \max(\norm{x}_p, \norm{y}_p)$ \cite{bib:analysis:volovich}
\end{enumerate}

Заметим, что норма $\norm{x}_p$ может принимать лишь счетное число значений $p ^ {-ord_pn}$.

Также, норма $\norm{x}_p$ определяет ультраметрику на $\mathbb {Q}$. Данная норма неархимедова, так как $\norm{nx}_p \le \norm{x}_p \forall n \in \mathbb {Q}_{+}$.

\begin{thethm}
	Нормы $\norm{.}$ и $\norm{.}_p$ $\forall p = 2, 3, \dots$ исчерпывают все нетривиальные неэквивалентные нормы поля рациональных чисел $\mathbb {Q}$.
\end{thethm}


\subsection{Пространство p-адических чисел $\mathbb {Q}_p$}

\begin{defn}
Пополнение поля $\mathbb {Q}$ по $p$-адической норме образует поле $\mathbb {Q}_p$ $p$-адических чисел. Поле $\mathbb {Q}_p$ аналогично полю $\mathbb {R} = \mathbb {Q}_{\infty}$ вещественных чисел, получаемых пополнение поля $\mathbb {Q}$ по норме $\norm{x}=\norm{x}_{\infty}$.
\end{defn}


\begin{defn}
Любое $p$-адическое число $x \ne 0$ однозначно представляется в каноническом виде

\begin{equation} \label{numbers:decomposition}
	x = p^{\gamma} \cdot (x_0 + x_1\cdot p + x_2 \cdot p^2 + \dots
\end{equation}

\noindent где $\gamma = \gamma(x) \in \mathbb {Z}$ и $x_j$ -- целые числа такие, что $0 \le x_j \le p-1$, $x_0 > 0,$ \linebreak $(j=0,1,\dots)$. 
\end{defn}

Представление \eqref{numbers:decomposition} аналогично разложению любого вещественного числа $x$ в бесконечную десятичную дробь:
\begin{equation*}
\begin{aligned}
	x=\pm10^\gamma \cdot (x_0 + x_1 \cdot 10^{-1} + x_2 \cdot 10^{-2} + \dots),\\
	\gamma \in \mathbb {Z}, x_j = 0, 1, \dots, 9, x_0 > 0,
\end{aligned}
\end{equation*}

\noindent и доказывается аналогично.

\begin{proposition}
Пусть $\alpha=p^m(a_0+a_1p+\cdots +a_np^n+\dots)$, где $0 \le a_i \le p$, $a_0 \neq 0$, - $p$ - адическое число. Противоположным к нему является число $- \alpha=\beta=p^m(b_0+b_1p+\cdots+b_np^n+\dots)$, где $b_0=p-a_0$ и $b_i=p-1-a_i$, при $i \geq 0$.
\end{proposition}\label{adic:pros:minus}

Помимо разложения, представление \eqref{numbers:decomposition} дает рациональные числа тогда и только тогда, когда, начиная с некоторого номера числа $x_j, j=0,1,\dots$ образуют периодическую последовательность.

\begin{defn}
Поле $\mathbb {Q}_p$ является коммутативно-ассоциативной группой по сложению;
\end{defn}

\begin{defn}
Поле $\mathbb {Q}_p^*=\mathbb {Q}_p \setminus \{0\}$ является коммутативно-ассоциативной группой по умножению;
\end{defn}

\begin{defn}
Поле $\mathbb {Q}_p^*$ называется мультипликативной группой поля $\mathbb {Q}_p$\cite{bib:analysis:baker};
\end{defn}

\begin{defn}
$p$-адические числа $x$, для которых $\norm{x}_p \le 1$ (т.e. $\gamma(x) \ge 0$ или $\{x\}_p=0$), называются целыми $p$-адическими числами, и их множество обозначается $\mathbb {Z}_p$. Множество $\mathbb {Z}_p$ является подкольцом кольца $\mathbb {Q}_p$; $\mathbb {Z}_+$ плотно в $\mathbb {Z}_p$. Целые числа $x \in \mathbb {Z}_p$, для которых $\norm{x}_p=1$, называютсяются единицами в $\mathbb {Z}_p$. \cite{bib:analysis:vladimirov}
\end{defn}

Совокупность элементов $x$ из $\mathbb {Z}_p$, для которых $\norm{x}_p < 1$ (т.e. $\gamma(x) \ge 0$ или $\norm{x}_p \le \frac{1}{p}$) образуют главный идеал кольца $\mathbb {Z}_p$; Данный идеал имеет вид $p\mathbb {Z}_p$. Поле вычетов $\mathbb {Z}_p \setminus p\mathbb {Z}_p$ состоит из $p$ элементов. В мультипликативной группе поля $\mathbb {Z}_p \setminus p\mathbb {Z}_p$ существует единица $\eta \ne 1$ порядка $p-1$ такая, что элементы $0, \eta, \eta^2, \dots, \eta^{p-1} = 1$ образуют полный набор представителей классов вычетов поля $\mathbb {Z}_p \setminus p\mathbb {Z}_p$.

В силу свойств $p$-адической нормы норма в поле $\mathbb {Q}_p$ удовлетворяет неравенству треугольника:
$$\norm{x + y}_p \le \max(\norm{x}_p, \norm{y}_p) \le \norm{x}_p + \norm{y}_p, x,y \in \mathbb {Q}_p.$$
\noindent Следовательно в $\mathbb {Q}_p$ можно ввести метрику:

\begin{equation}
	\rho (x,y)=\norm{x-y}_p.
\end{equation}

\noindent При этом $\mathbb {Q}_p$ становится полным метрическим пространством. Из представления \eqref{numbers:decomposition} следует сепарабельность $\mathbb {Q}_p$.  

\begin{defn}
$B_{\gamma}(a)$ -- круг радиуса $p^{\gamma^p}$ с центром в точке $a \in \mathbb {Q}_p$:
\begin{equation}
	B_\gamma(a) = \bigg\{x: \norm{x-a}_p \le p^{\gamma} \bigg\}, \gamma \in \mathbb {Z}
\end{equation}
\end{defn}

\begin{defn}
$S_{\gamma}(a)$ -- граница радиуса $p^{\gamma^p}$.
\begin{equation}
	S_\gamma(a) = \bigg\{x: \norm{x-a}_p = p^{\gamma} \bigg\}, \gamma \in \mathbb {Z}
\end{equation}
\end{defn}

\begin{lemma}
Если $b \in B_{\gamma}(a)$, то $B_{\gamma}(b)=B_{\gamma}(a)$.
\end{lemma}

\begin{note}
Круг $B_{\gamma}(a)$ и окружность $S_{\gamma}(a)$ -- открыто-замкнутые множества в $\mathbb {Q}_p$.
\end{note}

\begin{note}
Всякая точка круга $B_{\gamma}(a)$ является его центром.
\end{note}

\begin{note}
Любые два круга в $\mathbb {Q}_p$ либо не имеют общих точек, либо один содержится в другом.
\end{note}

\begin{note}
Всякое открытое множество в $\mathbb {Q}_p$ есть объединение не более чем счетного числа кругов без общих точек.
\end{note}

\begin{lemma} \label{lemma:2}
Если множество $M \subset \mathbb {Q}_p$ содержит две различные точки $a$ и $b$, $a \ne b$, то его можно представить в виде объединения непересекающихся открыто-замкнутых (в $M$) множеств $M_1, M_2$ таких, что $a \in M_1, b \in M_2$.
\end{lemma}

Лемма \eqref{lemma:2} утверждает, что всякое множество пространства $\mathbb {Q}_p$, состоящее из более чем одной точки, несвязно. Другими словами, связная компонента любой точки совпадает с самой точкой. Из этого следует, что $\mathbb {Q}_p$ является вполне несвязным пространством.

Если рассматривать лемму для случая, когда множество $M$ состоит только из двух точек $a$ и $b$, убеждаемся, что существует непересекающиеся окрестности этих точек. Из этого можно сделать вывод, что пространство $\mathbb {Q}_p$ хаусдорфово.

\begin{lemma}
Для того чтобы множество $K \subset \mathbb {Q}_p$ было компактом, необходимо и достаточно, чтобы оно было замкнутым и ограниченным в $\mathbb {Q}_p$
\end{lemma}

\begin{note}
Всякий круг $B_{\gamma}(a)$ является и окружность $S_{\gamma}(a)$ компакты.
\end{note}

\begin{note}
Пространство $\mathbb {Q}_p$ локально компактное.
\end{note}

\begin{note}
Всякий компакт можно покрыть конечным числом кругов фиксированного радиуса без общих точек.
\end{note}

\begin{note}
В пространстве $\mathbb {Q}_p$ справедлива лемма Гейне-Бореля: из каждого бесконечного покрытия компакта $K$ можно выбрать конечное покрытие $K$.
\end{note}

\begin{thethm}
Размерность пространства $\mathbb {Q}_p$ равна $0$.
\end{thethm}

\subsection{p-Адический анализ в $\mathbb {Z}_p$}

Так как компакт $\mathbb {Z}_p$ есть пополнение множества $\mathbb {N}_0$ по метрике \linebreak ${d_p(x,y)=\norm{x-y}_p}$, то любое число из $\mathbb {Z}_p$ есть предел последовательности чисел из $\mathbb {N}_0$.

\begin{defn}
$p$-адическое целое $z$ является пределом последовательности $\{z_i\}^{\infty}_{i=0}$, если если для любого $\epsilon > 0$ найдется $N$ такое, что $\norm{z_i-z}_p < \epsilon$ как только $i>N$. \cite{bib:analysis:anashin}
\end{defn}

\begin{defn}
$p$-адическое целое $z$ есть предел последовательности $\{z_i\}^{\infty}_{i=0}$, если для любого (достаточно большого) положительного рационального целого $K$ найдется $N$ такое, что ${z_i \equiv z \pmod p^K}$ при всех $i>N$. \cite{bib:analysis:anashin}
\end{defn}

\begin{note}
По определению $p$-адической метрики $\norm{z_i-z}_p \le p^{-K}$ тогда и только тогда, когда $z_i \equiv z \pmod p^K$. \cite{bib:analysis:anashin}
\end{note}

\begin{defn}
Функция $f:\mathbb {Z}_p \rightarrow \mathbb {Z}_p$ называется непрерывной в точке $z \in \mathbb {Z}_p$, если для любого (достаточно большого) положительного рационального целого $M$ найдется положительное рациональное целое $L$ такое, что ${f(x) \equiv f(z) \pmod p^M}$ как только $x \equiv z \pmod{p^L}$. \cite{bib:analysis:anashin}
\end{defn}

\begin{defn}
Функция $f$ называется равномерно непрерывной на $\mathbb {Z}_p$, если $f$ непрерывна в каждой точке $z \in \mathbb {Z}_p$, и $L$ зависит только от $M$ и не зависит от $z$.\cite{bib:analysis:ciocan}
\end{defn}


\begin{defn}
Функция $f:\mathbb {Z}_p \rightarrow \mathbb {Z}_p$ называется дифференцируемой в точке $z \in \mathbb {Z}_p$, если существует $p$-адическое число $f'(x) \in \mathbb {Q}_p$ такое, что для любого $M \in \mathbb {N}$ справедливо
\begin{equation} \label{derivative:1}
	\norm{\frac{f(x+h)-f(x)}{h} - f'(x)}_p \le \frac{1}{p^M},
\end{equation}

\noindent если $h$ достаточно мало, т.e. когда $\norm{h}_p \le p^{-K}$, где $K=K(M)$ достаточно велико.
\end{defn}

\begin{defn}
Функция $f$ называется равномерно дифференцируемой (на $\mathbb {Z}_p$), если неравенство \eqref{derivative:1} выполняется одновременно для всех $x \in \mathbb {Z}_p$ как только $h$ достаточно мало. \cite{bib:analysis:anashin:en}
\end{defn}

\begin{lemma}
Если совместимая функция $f:\mathbb {Z}_p \rightarrow \mathbb {Z}_p$ дифференцируема в точке $x \in \mathbb {Z}_p$, то $f'(x) \in \mathbb {Z}_p$.
\end{lemma}

\begin{defn}
Функция $f:\mathbb {Z}_p \rightarrow \mathbb {Z}_p$ называется дифференцируемой в точке $x \in \mathbb {Z}_p$, если существует $p$-адическое число $f'(x) \in \mathbb {Q}_p$ такое, что для любого $M \in \mathbb {N}$ справедливо
\begin{equation} \label{derivative:2}
	f(x+h) \equiv f(x) + h \cdot f'(x) \pmod p^{M + ord_p h}
\end{equation}
\end{defn}

\begin{defn}
Функция $f$ называется равномерно дифференцируемой (на $\mathbb {Z}_p$), если неравенство \eqref{derivative:2} выполняется одновременно для всех $x \in \mathbb {Z}_p$ как только $h$ достаточно мало, т.e. когда $ord_p h \ge K=K(M)$ для достаточно большого $K \in \mathbb {N}$.
\end{defn}

\begin{note}
Правила дифференцирования не зависят от метрики: для вычисления производных суммы, частного и сложной функции в $p$-адическом анализе используются те же формулы, что и в действительном.
\end{note}

\begin{note}
Между действительным и $p$-адическим анализом существует резкое различие например в том, что в и в том, и в в другом случае производная константы равна $0$, однако в $p$-адическом анализе в отличии от действительного равенство нулю производной некоторой функции не означает, что эта функция константа.
\end{note}




% код гензеля
% примеры сложения столбиком и другие операции

\section{Представления $p$-адических чисел и арифметические операции}

\subsection{Представление рациональных чисел в $p$-адической форме}
Пусть $\alpha=\frac{c}{d}$ - рациональное число. Покажем, как найти его $p$-адическое представление. Прежде всего, рассмотрим случай, когда ни $c$, ни $d$ не делятся на $p$. В этом случае $\alpha$ является единицей в поле $p$-адических чисел и может быть записано в виде $\alpha=a_0+a_1p+\cdots+a_np^n+\dots $, где $0 \textless a_0 \textless p$. Значение $a_0$ определяется условием $p \mid (a_0d-c)$ однозначно, поскольку кольцо вычетов по простому модулю является полем и деление на ненулевой элемент в поле всегда возможно и однозначно. Пусть $c-a_0d=c_1p$, $c1 \in \mathbb{Z}$. Тогда $\alpha=a_0+p\frac{c_1}{d}$ и коэффициент $a_1$ однозначно определяется условием $p \mid (a_1d-c_1)$ (он может быть равен нулю). Продолжая этот процесс, мы можем найти любое конечное число цифр в $p$-адическом представлении числа $\alpha$. Для представления чисел, не являющихся $p$-адическими единицами, нужно воспользоваться следующей теоремой:

\begin{thethm}
Всякое отличное от нуля $p$-адическое число $\xi$ однозначно представляется в виде

$$
\xi=p^m(a_0+a_1p^1+\cdots+a_np^n+\dots)
$$

\noindent где $m=\nu_p(\xi)$, $1 \le a_0 \le p-1$, $0 \le a_n \le p-1$$(n=1,2,\dots)$.
\end{thethm}

По аналогии с представлением вещественных чисел в виде бесконечных десятичных дробей, для $p$-адических чисел справедливо следующее утверждение

\begin{proposition}
любое рациональное число может быть представлено в виде переодического $p$-адического числа. Всякое переодическое $p$-адическое число представляет некоторое рациональное число.
\end{proposition}

\subsection{Арифметические операции}

Сложение и умножение $p$-адических чисел выполняется аналогично сложению и умножению десятичных дробей с тем отличием, что цифры складываются или умножаются не справа налево, а слева направо и переносы осуществляются в следующую позицию направо.

\begin{exmp}
Сложить $\frac{2}{3}$ и $\frac{5}{6}$ в $\mathbb{Z}_5$.

\noindent $5$-адическое разложение слагаемых имеет вид

$$
\frac{2}{3}=.4131313\dots
$$
$$
\frac{5}{6}=.0140404\dots
$$
Выполняя сложение, получим

$$
\begin{tabular}{cccccccccccc}
& + & . & 4\; & 1\; & 3\; & 1\; & 3 & 1 & 3 & \dots \\
& = & . & 0\; & 1\; & 4\; & 0\; & 4 & 0 & 4 & \dots \\
\hline
& = & . & 4\; & 2\; & 2\; & 2\; & 2 & 2 & 2 & \dots
\end{tabular}
$$

\noindent Видно, что $5$-адическое представление числа $\frac{2}{3} + \frac{5}{6}=\frac{3}{2}=.4222222\dots$
\end{exmp}

\begin{exmp}
Перемножить $\frac{2}{3}$ и $\frac{5}{6}$ в $\mathbb{Z}_5$.

\noindent $5$-адическое разложение сомножителей имеет вид

$$
\frac{2}{3}=.4131313\dots
$$
$$
\frac{5}{6}=.0140404\dots
$$
Выполняя умножение, получим

$$
\begin{tabular}{ccccccccccccccccc}
& + & . & 4\; & 1\; & 3\; & 1\; & 3 & 1 & 3 & 1 & 3 & 1 & 3 & \dots \\
& = & . & 1\; & 4\; & 0 & 4\; & 0\; & 4 & 0 & 4 & 0 & 4 & 0 & \dots \\
\hline
& & & 4\; & 1\; & 3\; & 1\; & 3 & 1 & 3 & 1 & 3 & 1 & 3 & \dots \\
& & & & 1\; & 2\; & 3\; & 1 & 3 & 1 & 3 & 1 & 3 & 1 & \dots \\
& & & & & & 1\; & 2\; & 3\; & 1 & 3 & 1 & 3 & 1 & \dots \\
& & & & & & & & 1\; & 2\; & 3\; & 1 & 3 & 1 & \dots \\
& & & & & & & & & & 1\; & 2\; & 3\; & 1 &  \dots \\
& & & & & & & & & & & & 1\; & 2\; & \dots \\
\hline
& = & . & 4\; & 2\; & 0 & 1\; & 2\; & 4 & 3 & 2 & 0 & 1 & 2 & \dots \\
\end{tabular}
$$

\noindent Видно, что $5$-адическое представление числа $\frac{2}{3} * \frac{5}{6}=\frac{1}{9}=.4201243201243\dots$
\end{exmp}


Вычитание $p$-адических чисел рекомендуется выполнять в два этапа. Сначала, воспользовавшись предположением \ref{adic:pros:minus}, свести задачу к сложению двух $p$-адических чисел, а затем выполнить это сложение.

\begin{exmp}
Вычесть $\frac{5}{6}$ и $\frac{2}{3}$ в $\mathbb{Z}_5$.

\noindent $5$-адическое представление отрицательного операнда имеет вид

$$
-\frac{5}{6}=.040404\dots
$$

\noindent Выполняя вычитание, получим

$$
\begin{tabular}{cccccccccccc}
& - & . & 4\; & 1\; & 3\; & 1\; & 3 & 1 & 3 & \dots \\
& = & . & 0 \; & 4\; & 0\; & 4 & 0 & 4 & 0 & \dots \\
\hline
& = & . & 4\; & 0\; & 4\; & 0\; & 4 & 0 & 4 & \dots
\end{tabular}
$$

\noindent Видно, что $5$-адическое представление числа $\frac{2}{3} - \frac{5}{6}=-\frac{1}{6}=.04040404\dots$
\end{exmp}


Деление $p$-адических чисел выполняется во-многом аналогично делению столбиком десятичных дробей. Однако, кроме особенности выполнения операций слева направо, отметим еще две: во-первых, вычитание заменяется домножением вычитаемого на $-1$ и последующим сложением, а самое главное, деление является алгоритмическим в том смысле, что первая цифра частного однозначно определяется первыми цифрами делимого и делителя.

\begin{exmp}
Разделить $\frac{2}{3}$ и $\frac{1}{12}$ в $\mathbb{Z}_5$.

\noindent $5$-адическое разложение делимого и делителя имеет вид

$$
\frac{2}{3}=.4131313\dots
$$

$$
\frac{1}{12}=.3424242\dots
$$

\noindent Первой цифрой знаменателя является $3$, обратный к ней элемент в $\mathbb{Z}_5$ - это $2$, т.e. $3^{-1} \equiv 2 \pmod 5$. Следовательно, первая цифра частного равна $4*2 \equiv 3 \pmod 5$. Умножая делитель на $3$, а затем на $-1$, получаем $.111111\dots$. Теперь можем выполнить первый шаг деления столбиком.

$$
\arraycolsep=0.01em
\begin{array}{rrrrrrrr@{\,}r|l}
.&4&1&3&1&3&1&\dots&&\,.3424241\dots\\
\cline{10-10}
&1&1&1&1&1&1&\dots&&\,.3\\
\cline{1-6}
&&3&4&2&4&2&\dots
\end{array}
$$

\noindent Очевидно, что следующей цифрой частного является $1$. Продолжаем деление 

$$
\arraycolsep=0.01em
\begin{array}{rrrrrrrr@{\,}r|l}
.&4&1&3&1&3&1&\dots&&\,.3424241\dots\\
\cline{10-10}
&1&1&1&1&1&1&\dots&&\,.3\\
\cline{1-6}
&&3&4&2&4&2&\dots \\
&&2&0&2&0&2&\dots \\
\cline{1-6}
&&0&0&0&0&0&\dots
\end{array}
$$

в остатке получили $0$, значит деление завершено. В частном мы получили целое число $8$. Легко убедиться, что $\frac{2}{3} \div \frac{1}{12}=8$ и $8=.310000\dots$ в $\mathbb{Z}_5$.

Очевидно, что в общем случае мы ни на каком шаге не получим в остатке $0$. Деление можно продолжать бесконечно. Если делимое и делитель - рациональные числа, что естественно остановиться, как только найдем период частного (который существует, поскольку в этом случае частное также является рациональным числом).

$$
\begin{tabular}{cccccccccccc}
& - & . & 4\; & 1\; & 3\; & 1\; & 3 & 1 & 3 & \dots \\
& = & . & 0 \; & 4\; & 0\; & 4 & 0 & 4 & 0 & \dots \\
\hline
& = & . & 4\; & 0\; & 4\; & 0\; & 4 & 0 & 4 & \dots
\end{tabular}
$$

\noindent Видно, что $5$-адическое представление числа $\frac{2}{3} - \frac{5}{6}=-\frac{1}{6}=.04040404\dots$
\end{exmp}



\subsection{Код Гензеля}
По аналогии с приближением вещественных чисел конечными дробями с фиксированным числом знаков после десятичной (или двоичной, или восьмеричной и т.д.) точки можно рассматривать приближения $p$-адических чисел конечными отрезками их $p$-адического представления с фиксированным числом знаков после $p$-адической точки.

\begin{defn}

Пусть $p$-простое число, $r$-натуральное число и $\alpha=\sum\limits^{\infty}_{i=m} a_ip^i$ - $p$-адическое число ($0 \le a_i \le p, a_m \neq 0$). Кодом Гензеля $p$-адического числа $\alpha$ назовем $p$-адическое представление числа $\sum\limits_{i=m}^{r}a_ip^i$. Для кодов гензеля будем использовать обозначение $H(p,r,\alpha)$, явно содержащее числа $p$, $r$ и $\alpha$. Например, $H(5,4,\frac{1}{3})=.2313$.

\end{defn}

Легко видеть, что если $\alpha$ - рациональное число, то его код Гензеля $\beta=H(p,r\alpha)$ - это такое целое число или несократимая дробь со знаменателем вида $p^k$ для некоторого натурального $k$, что $\alpha-\beta$ представляется несократимой дробью, числитель которой делится на $p^r$. Это условие можно так же обозначить как $\alpha - \beta \equiv 0 \pmod {p^r}$.

Такие коды Гензеля соответствуют представлению вещественных чисел с фиксированной точкой, когда фиксируется абсолютная погрешность представления. Однако существенным отличием $p$-адической арифметики является то, что при сложнении или вычитании чисел абсолютная погрешность не накапливается.

Так же можно ввести понятие кода Гензеля с плавающей точкой. Пусть $\alpha$ - $p$-адическое кольцо. Представим его в виде $\alpha=p^n\epsilon$, где $\epsilon$ - единица кольца целых $p$-адических чисел. В этом случае нормализованным кодом Гензеля с плавающей точкой $\hat H (p, r, \alpha)$ числа $\alpha$ назовем пару $(m_{\alpha}e_{\alpha})$, где $m_\alpha = H(p,r,\alpha)$ и $e_\alpha=n$. Мы назовем $m_\alpha$ мантиссой, а $e_\alpha$ - показателем числа $\alpha$.

Коды Гензеля с плавающей точкой соответствуют представлению вещественных чисел с фиксированной относительной точностью. Снова отметим тот факт, что при умножении кодов Гензеля с плавающей точкой относительная погрешность не накапливается, как это имеет место при умножении вещественных чисел.


\subsection{Примеры операций с кодом Гензеля}

\begin{exmp}
Получить код Гензеля для разности $\frac{3}{4}$ и $\frac{3}{2}$ в $\mathbb{Z}_5$.

\noindent Код Гензеля для операндов имеет вид

$$H(5,4, \frac{3}{4})=(.\; 2\; 1\; 1\; 1,\; 0)$$

$$H(5,4, \frac{3}{2})=(.\; 4\; 2\; 2\; 2,\; 0)$$


\noindent Произведем вычитание

$$
\begin{tabular}{ccccccccccc}
& - & .\; & 2\; & 1\; & 1\; & 1\; & ,\; & 0\; &  \\
& = & .\; & 4 \; & 2\; & 2\; & 2\; & ,\; & 0\; &  \\
\hline
& = & .\; & 3\; & 3\; & 3\; & 3\; & ,\; & 0\; &
\end{tabular}
$$


\noindent Таким образом результатом будет код Гензеля $(.\; 3\; 3\; 3\; 3,\; 0)$, который представляет собой рациональное число $-\frac{3}{4}$.
\end{exmp}


\begin{exmp}
Получить код Гензеля для суммы $\frac{3}{10}$ и $\frac{1}{2}$ в $\mathbb{Z}_5$.

\noindent Код Гензеля для операндов имеет вид

$$H(5,4, \frac{3}{10})=(.\; 4\; 2\; 2\; 2,\; -1)$$

$$H(5,4, \frac{1}{2})=(.\; 3\; 2\; 2\; 2,\; 0)$$


\noindent Поскольку показатели отличаются, мы должны нормализовать код, который имеет больший показатель

$$ 
(.\; 3\; 2\; 2\; 2,\; 0) \rightarrow (.\; 0 \; 3\; 2\; 2\; ,\; -1)
$$

\noindent Теперь мы можем произвести сложение

$$
\begin{tabular}{ccccccccccc}
& - & .\; & 4\; & 2\; & 2\; & 2\; & ,\; & -1\; &  \\
& = & .\; & 0\; & 3\; & 2\; & 2\; & ,\; & -1\; &  \\
\hline
& = & .\; & 4\; & 0\; & 0\; & 0\; & ,\; & -1\; &
\end{tabular}
$$


\noindent Таким образом результатом будет код Гензеля $(.\; 4\; 0\; 0\; 0,\; -1)$, который представляет собой рациональное число $\frac{4}{5}$.
\end{exmp}

\begin{exmp}
Получить код Гензеля для произведения $\frac{4}{5}$ и $\frac{5}{2}$ в $\mathbb{Z}_5$.

\noindent Код Гензеля для операндов имеет вид

$$H(5,4, \frac{4}{5})=(.\; 3\; 3\; 1\; 3,\; -1)$$

$$H(5,4, \frac{5}{2})=(.\; 3\; 2\; 2\; 2,\; 1)$$

\noindent Теперь мы можем произвести умножение

$$
\begin{tabular}{cccccccccc}
& + & .\; & 3\; & 3\; & 1\; & 3\; & ,\; & -1\; \\
& = & .\; & 3\; & 2\; & 2\; & 2\; & ,\; & 1\; \\
\hline
& & & 4\; & 0\; & 0\; & 0\; & & & \\
& & & & 1\; & 2\; & 3\; & & & \\
& & & & & 1\; & 2\; & & & \\
& & & & & & 1\; & & & \\
\hline
& = & . & 4\; & 1\; & 3\; & 1\; & ,\; & 0\; &
\end{tabular}
$$


\noindent Таким образом результатом будет код Гензеля $(.\; 4\; 1\; 3\; 1,\; 0)$, который представляет собой рациональное число $\frac{2}{3}$.
\end{exmp}



%kke 
\begin{exmp}
Получить код Гензеля для частного $\frac{3}{4}$ и $\frac{6}{5}$ в $\mathbb{Z}_5$.

\noindent Код Гензеля для операндов имеет вид

$$H(5,4, \frac{3}{4})=(.\; 2\; 1\; 1\; 1,\; 0)$$

$$H(5,4, \frac{6}{5})=(.\; 1\; 1\; 0\; 0,\; -1)$$

\noindent Теперь мы можем произвести деление и получим код Гензеля $(.\; 2\; 4\; 1\; 3,\; 1)$, который представляет собой рациональное число $\frac{5}{8}$.
\end{exmp}


\section{Разработка однопоточной библиотеки для работы с $p$-адической арифметикой}

Библиотека для работы с $p$-адической арифметикой будет состоять из набора Python модулей, которые будут представлять из себя единый пакет для пакетного менеджера pip. Данный подход позволит в дальнейшем использовать библиотеку без каких-либо сложностей путем добавления ее в файл requirements.txt. Однопоточная библиотека будет содержать следующие модули:

\begin{itemize}
 
  \item Модуль для работы с $p$-адическими числами. Содержит базовые операции и представляет основные интерфейсы, такие как ввод и вывод данных чисел.
  \item Модуль для работы с СЛАУ. Предоставляет базовые операции над матрицами.
 
\end{itemize}


\subsection{Описание типов данных}

Типы данных, которые используются в библиотеке, представляются классами, так как это является более удобным форматом для работы с объектами.

Для описания $p$-адических чисел был реализован Python класс \\ |PAdic(object)|, который предоставляет следующие возможности:

 \begin{itemize}
 
  \item Сложение $p$-адических чисел. Данная операция реализована с помощью метода |__add__|, этот метод является одним из многих т.н. |Magic Methods|, которые позволяют пользователю работать с математическими объектами более естественным образом, а именно использовать знак |+| вместо явного вызова функции которое пользователю может быть неизвестно без подробного ознакомления с библиотекой. Сам метод |__add__| является в свою очередь все лишь оберткой над методом |add_by_offset|, который представляет собой сложение $p$-адических чисел представленных кодом Гензеля.
  \item Вычитание $p$-адических чисел. Данный метод реализован аналогично сложению за исключением того, что использует метод |__sub__|, который в свою очередь является оберткой над методом |subtract_by_offset|.
  \item Вычисление числа со знаком минус. Данная операция реализована с помощью метода |__neg__|. В коде это может быть использовано как: |a = -b|, где |b| это $p$-адическое число.
  \item Вычисление числа со знаком плюс. Данная операция реализована с помощью метода |__pos__|. В коде это может быть использовано как: |a = +b|, где |b| это $p$-адическое число. Данная операция обычно реализуется для симметрии, поскольку унарный минус является оператором, унарный плюс тоже должен быть.
  \item Умножение $p$-адического чисела на целое число. Для данной операции реализован метод |multiply_to_integer|, который является результатом произведения операндов.
  \item Умножение $p$-адических чисел. Для данной операции был использован метод |__mul__|, который нужен для возможности использования знака |*| при работе с числами и представляет собой полноценный метод реализующий $p$-адическое умножение чисел представленных кодом Гензеля.
  \item Деление $p$-адических чисел. Данная операция реализована с помощью метода |__truediv__|, который нужен для возможности использования знака |/| при работе с числами и представляет собой полноценный метод реализующий $p$-адическое умножение чисел представленных кодом Гензеля.
  \item Вычисления порядка $p$-адического числа. Операция реализована методом |calculate_order|. Так как метод статический, то может быть вызван для любого объекта класса из вне с помощью вызова метода \\ |PAdic.calculate_order(a)|, где |a| это $p$-адическое число.
  \item Получение порядка $p$-адического числа. Операция реализована методом |get_order|.
  \item Вывод $p$-адического числа в человеко-читаемом формате. Реализовано с помощью метода |__str__|. Поскольку функция |print| использует именно функцию |str()| для вывода объекта на экран, то определение метода |__str__| позволит выводить объекты на экран удобным способом: при помощи |print()|.
  \item Вывод $p$-адического числа в виде объекта класса |PAdic|. Реализовано с помощью метода |__repr__|, который возвращает строку с описанием объекта, которое может быть воспринято итерпретатором языка Python.


  %\item find_multiplier
  %\item do_eratosthene_sieve
  %\item check_for_base_equality
  %\item check_for_prime
\end{itemize}


Для работы с $p$-адическими матрицами был реализован Python класс \\ |Matrix(object)|, который предоставляет следующие возможности:


\subsection{Описание арифметических операций}
сложение, эквивалентность, вычитание, деление

\begin{itemize}
\item |__str__|
\item |__repl__|
\item |__getitem__|
\item |__setitem__|
\item |get_rank|
\item |reset|
\item |transpose|
\item |get_transpose|
\item |__add__|
\item |__iadd__|
\item |__sub__|
\item |__isub__|
\item |__mul__|
\item |__imul__|
\item |__eq__|
\item |make_matrix|
\item |make_random|
\item |make_zero|
\item |make_id|


	
\end{itemize}



\subsection{Прикладные примеры использования библиотеки}
 
 
\section{Разработка многопоточной библиотеки для работы с $p$-адической арифметикой}
\subsection{Описание алгоритмов для арифметических операций}
\subsection{Сравнение производительности однопоточной и многопоточной библиотеки} 


\section{Сравнение производительности классических и $p$-адических арифметических операций}
\subsection{Решение СЛАУ}
\subsection{Решение ОДУ}
\subsection{Вычисление $e^{Ax}$}


 

\conclusion
TODO

\bibliographystyle{biblio/ugost}
\bibliography{biblio/biblio}

\appendix

\section{Нумеруемые объекты в приложении}

тут будет код


\end{document}