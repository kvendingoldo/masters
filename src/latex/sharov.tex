\documentclass[master, och, diploma, times]{sty/SCWorks}

\usepackage[T2A]{fontenc}
\usepackage[utf8]{inputenc}
\usepackage{graphicx}

\usepackage[sort,compress]{cite}
\usepackage{amsmath}
\usepackage{amssymb}
\usepackage{amsthm}
\usepackage{fancyvrb}
\usepackage{longtable}
\usepackage{array}
\usepackage[english,russian]{babel}

\usepackage[colorlinks=true]{hyperref}

\begin{document}

% Кафедра (в родительном падеже)
\chair{дискретной математики}
% Тема работы
\title{Приложение p-адической арифметики к задачам компьютерной алгебры}
% Курс
\course{2}
% Группа
\group{217}
% Специальность/направление код - наименование
\napravlenie{09.03.01 "--- Информатика и вычислительная техника}

% Фамилия, имя, отчество в родительном падеже
\author{Шарова Александра Вадимовича}
% Заведующий кафедрой
\chtitle{к.\,ф.-м.\,н.} % степень, звание
\chname{Л.\,Б.\,Тяпаев}
%Научный руководитель (для реферата преподаватель проверяющий работу)
\satitle{к.\,ф.-м.\,н.} %должность, степень, звание
\saname{Л.\,Б.\,Тяпаев}
% Руководитель практики от организации (только для практики,
% для остальных типов работ не используется)
\patitle{к.\,ф.-м.\,н., доцент}
\paname{Д.\,Ю.\,Петров}

% Год выполнения отчета
\date{2020}

\maketitle
\tableofcontents

\abbreviations
\begin{enumerate}
\item TODO -- TODO
\end{enumerate}

\intro
тут будет введение

\section{глава 1}
TODO

\section{глава 2}
TODO

\section{глава 3}
TODO

\conclusion
TODO

\bibliographystyle{biblio/ugost}
\bibliography{biblio/biblio}

\appendix

\section{Нумеруемые объекты в приложении}

тут будет код


\end{document}