\documentclass[10pt,professionalfont,utf8,presentation,compress]{beamer}

\usepackage{sty/beamer}

\uselanguage{russian}
\languagepath{russian}

\usepackage{multicol} 		% Несколько колонок
\graphicspath{{images/}}  	% Папка с картинками

\usepackage{siunitx}
\usepackage{newtxtext,newtxmath}
\usepackage{bm}

\setbeamerfont{author}{size=\fontsize{11pt}{12pt}}
\setbeamerfont{institute}{size=\fontsize{10pt}{11pt}}

%\usetheme{Madrid}


% информация для титульника и для подписей слайдов
\title[Приложения $p$-адической арифметики]
{Приложение $p$-адической арифметики к задачам компьютерной алгебры}
\author{А.~В.~Шаров}
\institute{{Саратовский государственный университет} \\
    им.~Н.~Г.~Чернышевского \\[5pt]
Кафедра дискретной математики\\[5pt]
Научный руководитель: к.~ф.-м.~н., доцент Тяпаев~Л.~Б.
}
\date{29 Мая 2020 г.}


\begin{document}

\frame[plain]{\titlepage}

\begin{frame}
\frametitle{$p$-адические числа}
	some text
\end{frame}


\begin{frame}
\frametitle{Представления $p$-адических чисел}
	some text
\end{frame}

\begin{frame}
\frametitle{Код Гензеля}
	some text
\end{frame}


\begin{frame}
\frametitle{Программная реализация}
	some text
\end{frame}

\begin{frame}
\frametitle{Сравнение производительности: решение СЛАУ}
	some text
\end{frame}

\begin{frame}
\frametitle{Сравнение производительности: вычисление собственных значений и собственных векторов}
	some text
\end{frame}

\begin{frame}
\frametitle{Сравнение производительности: решение ОДУ}
	some text
\end{frame}

\begin{frame}
\frametitle{Сравнение производительности: вычисление СЛАУ}
	some text
\end{frame}


\begin{frame}[c]
\begin{center}
\frametitle{\LARGE Спасибо за внимание!}

{\LARGE \inserttitle}

\bigskip\bigskip

{\large \insertauthor}

\bigskip\bigskip

{\insertinstitute}

\bigskip\bigskip

{\large \insertdate}
\end{center}
\end{frame}

\end{document}
